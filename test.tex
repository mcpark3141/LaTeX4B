\documentclass[a4paper, nobookmarks, subfigure, adjustmath, amsmath]{oblivoir}
% \usepackage{fapapersize}
	% \usefapapersize{*,*,30mm,*,35mm,*}
	% \pagestyle{myheadings}
	% \makeoddhead{myheadings}{}{}{주간보고 - 박명철, 2015년 6월 11일}
	% \makeoddfoot{myheadings}{}{\thepage}{}
\usepackage{xetexko-hanging}
\usepackage{mathptmx}
	\DeclareMathAlphabet{\mathcal}{OMS}{cmsy}{m}{n}
	\setmainfont{Times New Roman}
	% \setmainhangulfont{HCR Batang LVT}
\usepackage{graphicx, xcolor}
	\graphicspath{{figure/}}
	% \def\figurename{Fig.}
	\feetbelowfloat
\usepackage{subfigure}
\usepackage{mathrsfs,amsthm,mathtools}
	% \renewcommand\proofname{\mbox{증명}}
	\allowdisplaybreaks
\usepackage[shortlabels]{enumitem}			% Vertical space between items
	\setlist{noitemsep}
	\setlist{nolistsep}
%	\setlist[itemize]{leftmargin=*}
\usepackage[compatibility=true]{caption}
%\usepackage{url}
\usepackage[noadjust]{cite}
	\hypersetup{colorlinks=false,
				pdfborder={0 0 0},
				pdfstartview=FitH}
% \usepackage{jiwonlipsum}
\usepackage{autonum}












\title{\LaTeX 간단 설명서}
\author{mcpark}

\begin{document}
\maketitle
\begin{abstract}
논문 작성을 위해 주로 사용하는 도구는 \LaTeX이다.
이 문서에서는 \LaTeX을 이용해 논문을 작성할 수 있는 환경을 꾸리는 데까지 필요한 과정을 간략히 설명하고 논문 작성 중간에 필요한 \LaTeX의 기능들을 설명한다.
\end{abstract}

%%%%%%%%%%%%%%%%%%%%%%%%%%%%%%%%%%%%%%%%%%%%%%%%%%%%%%%%%%%%%%%%%%%%%%%
\section{설치}
GUI 테스트...


%%%%%%%%%%%%%%%%%%%%%%%%%%%%%%%%%%%%%%%%%%%%%%%%%%%%%%%%%%%%%%%%%%%%%%%
\section{간단한 문서 생성}
수정1
수정2

Revert 테스트

BranchTest

<<<<<<< HEAD
master branch에서 수정됨
=======
BranchTest2
>>>>>>> branchTest
%%%%%%%%%%%%%%%%%%%%%%%%%%%%%%%%%%%%%%%%%%%%%%%%%%%%%%%%%%%%%%%%%%%%%%%
\section{학회 논문 및 저널 논문 작성}


%%%%%%%%%%%%%%%%%%%%%%%%%%%%%%%%%%%%%%%%%%%%%%%%%%%%%%%%%%%%%%%%%%%%%%%
\section{학위논문 작성}
















\end{document}